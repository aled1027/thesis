\chapter{Chaining Garbled Circuits}

At the end of the previous chapter, we discussed OT-preprocessing in which some work is in done in an offline phase that speeds up the computation of the online phase.
We can also split the garbled circuits scheme into online/offline phase.
During the offline phase, the garbler generates and garbles the circuit, and sends it to the evaluator.
Then, when the garbler and evaluator acquire their inputs to the function that they want to compute, they exchange input labels - some via the online phase of OT-preprocessing - and the evaluator evaluates the garbled circuit.
Splitting the work into offline/online phases dramatically reduces the real world overhead required in performing secure computation.
Imagine that two banks integrate secure computation into their daily transactions.
At night when activity is low, their servers exchange many garbled circuits, and then during the day, they use the pre-exchanged garbled circuits to quickly do secure comptuation.
The computational requirements for the banks, when they compute the function, is only to exchange input labels and for the evaluator-bank to evaluate the garbled circuit.
The size of the garbled tables and the size of the function are both irrelevant to the computation.

With this more real world scenario, we have a new open question: how can we improve garbled circuits in the offline/online setting?
Let us narrow the question a bit more.
What are the problems with the garbled circuits in the offline/online setting?
Two problems immediately come to mind. 
First, the function that the parties want to perform is fixed by the computation done in the offline phase.
If the two banks want to compute a particular function, but that function was not turned into a circuit and exchanged the night before, then the banks are out of luck - they cannot perform that function in the online/offline model.
We refer to this problem as the flexibilty of functionality problem.
Second, we have the flexibility of inputs problem.
Garbled circuits have a fixed input size.
For some functions, this is not a problem as the extra inputs can simply be padded into $0$s, which solves the problem of flexibilty but causes the evaluator to compute a needlessly large circuit.
Imagine a setting of multiparty computation where multiple parties are voting.
In the online/offline setting, the size of the circuit is fixed, so there is some maximum number of parties that can participate in the election.
If more parties want to vote than predicted, then a party has to create a new function, perform with all parties again, and so on.
It would be great if there some garbled circuit method that could handle arbitarily large or small number of inputs.

This chapter discusses \textit{chaining} garbled circuits, a method which provides flexibility of functionality and flexibility of inputs.
The idea is based on stringing together many smaller garbled circuits in order to compute a larger function.
For example, many useful functions can be decomposed into multiplying matrices.
The garbler can generate a bunch of circuits that multiple matrices and exchange them during the offline phase.
At the beginning of the online phase, the garbler and evaluator agree on a function to compute, figure out how chain their matrix-multiplying-garbled-circuits in a way to compute their function, and do so.
They have know computed a function, which was totally determined at online time.
We might say that garbled circuits exchanged during the offline phase offer a \textit{class} of functions that the evaluator and garbler can compute.
The evaluator and garbler then pick a function from the class of functions during the online phase.

\section{The Random Oracle Model and Random Permutation Model}
In order to discuss chaining, we need to introduce a little more theory.
This section discusses two models in which we imagine cryptographic schemes.

The scheme that we have been working in is called the \textit{real world model} \al{What's it called?}
In the real world model, we assume that parties have access to a polynomial length string of random bits, preset at the beginning of their algorithm.
This string can be used however the parties wish, but it is a static access of randomness.

The \texit{random oracle model} is a relaxation of the real world model.
In the random oracle model, parties have a access to $H$, an oracle that on input of any value, outputs a perfectly random string. 

What is an oracle?
clearly not real - but essentially real?

The \textit{random permutation model} is a further relexation of the random oracle model.
In the random oracle model, parties

\section{Chaining}
\section{Security of Chaining}
\section{Single Communication Multiple Connections}
\section{Security of SCMC}
