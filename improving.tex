\chapter{Improving MPC}
\al{cite Mike Rosulek's talk and website}
\al{Mike's website has all the sources...}
\al{figure out why GRR3 is secure}
\al{Ask about notation: $A_0$ verse $W_i^0$}

The aim of this chapter is to outline new techniques that improve the performance of garbled circuits and discuss their costs and benfits.

We consider three metrics to understand the costs benefits of the new techniques: size of the garbled table, garbler-side computation and evaluator-side computation
\footnote{In this chapter, the term garbled table is used interchangeably with the fourth column of the garbled table.}.
In the classic scheme of garbled circuits, the garbler and evaluator communicate wire labels of input wires and the fourth coumn of the garbled table for each gate.
Communicating the input labels is an unavoidable cost; this information needs to be exchanged. 
However, many of the input labels are exchanged with oblivious transfer, discussed in chapter 1, which has been improved.
We discuss the improvement to oblivious transfer at the end of this chapter.

New techniques also reduce the size of the garbled table, the other infomration exchanged between the garbler and evaluator.
\footnote{In chapter 2, Alice was the garbler and Bob was the evaluator. For clarity in chapter 3 and beyond, the term \textit{garbler} to allude to the person who garbles the circuit, and the term \textit{evaluator} to describe the person who evaluates, or decrypts, the garbled circuit.}.
In the classical garbled circuit scheme, the evaluator sends all four rows of the garble table to the evaluator.
New techiniques, wherein the evaluator samples wire labels \textit{intelligently}, reduce the number of rows of the garbled table that are communicated.
Reducing the size of the garbled table is important, because it is a measure of the bandwidth requirement of the scheme.
The garbled table amount to most of the communication, and the number of garbled tables scales with the size of the circuit - that is, a larger circuit means more gates means more garbled tables.
Moreover, bandwidth is the most important factor in assessing the performance of a garbled circuit scheme.
This is because communication of the internet is slower than local computation.

We also consider garbler-side computatoin, which is the number of encryptions that the garbler performs in preparing a garbled table.
In the classical garbled circuit scheme, the garbler performs four encryptions; however, this number is not often reduced.
Typically, the new techniques increase or alter the computation that the garbler performs.

Finally, we consider evaluator-side computation, which is the number of decryptions the evaluator performs in ungarbling a gate - that is, finding the output wire label.
In the Classical garbled circuit scheme, the evaluator trial decrypts each row until one of the decryptions is successful.
Hence, on average, the evaluator performs 2.5 decryptions.
New techniques reduce the evaluator-side computation to require a single decryption, a good performance improvement.

Table \ref{tbl:gc-costs} shows an overview of the cost of all improvements made to garble circuits.
The table is split into three sections: size, eval cost and garble cost, corresponding to three metrics mentioned above.
Size is the number of rows in the garbled table.
Eval cost is number of encryptions that the evaluator performs per gate, and decryptions is the number of decryptions that the garbler performs per gate.
Because many of the techniques have different effects on XOR gates and AND gates, each section is divided into two columns showing the metrics for AND gates and XOR gates separately.

The goal of this chapter is primarily to explain each row in this table.
We will understand the technique, and its associated computational and bandwidth costs.
We start with the earliest technique and move chronological to the most recent improvement.

\begin{table}[t]
    \label{tbl:gc-costs}
    \centering
    \renewcommand{\arraystretch}{1.2}
    \begin{adjustbox}{width=1\textwidth}
        \begin{tabular}{|p{5cm}|c|c|c|c|c|c|c|}
            \hline
            \multirow{2}{5cm}{\centering \textbf{Garbled Circuit Improvement}} & 
            \multicolumn{2}{c|}{\textbf{Size (x$\lambda$)}} & 
            \multicolumn{2}{c|}{\textbf{Eval Cost}} & 
            \multicolumn{2}{c|}{\textbf{Garble Cost}} &
            \multirow{2}{3cm}{\centering \textbf{Assumption}} \\
            \cline{2-7}
            & \textbf{XOR} & \textbf{AND} & \textbf{XOR} & \textbf{AND}  & \textbf{XOR} & \textbf{AND} & \\
            \hline
            Classical & 4 & 4 & 1024 & $\mu$ s & a & b & a\\ \hline
            Point and Permute & TBD & TBD & TBD & $\mu$ s & a & b & a\\ \hline
            GRR3 & TBD & TBD & TBD & $\mu$ s  & a & b& a\\ \hline
            Free XOR & TBD & TBD & TBD & $\mu$ & a & bs& a  \\ \hline
            GRR2 XOR & TBD & TBD & TBD & $\mu$ & a & bs  & a\\ \hline
            FleXOR& TBD & TBD & TBD & $\mu$ & a & bs & a \\ \hline
            Half Gates & 2 & 0 & 2 & 0 & 4 & 0 & a \\ \hline
        \end{tabular}
    \end{adjustbox}
    \caption{Summary of Garbled Circuit Improvements. GRR3 stands for garbled row reduction 3 and GRR2 stands for garbled row reduction 2. \al{two halves make a whole paper}}
\end{table}

\section{Point and Permute}
\al{add to chapter 2 that rows need to be permuted!}

The \textit{point and permute} technique speeds up the evaluator's computation of the garbled table by removing the need to trial decrypt the ciphertexts; instead, the garbler subtly communicates which row of the garbled table to decrypt.
Recall that the garbled table is randomly permuted - that is, garbler randomly reorders the rows of the garbled table before sending the garbled table to the evaluator \footnote{This is required for security. See chapter 2 for more information}.
Upon receiving the circuit, the evaluator trial decrypts each row of the garbled table until a decryption succeeds \footnote{Recall that the decryption algorithm outputs a single bit indicating whether or not the decryption was successful. For more information, see chapter 1.}.

Point and permute enables the evaluator to bypass the trial decryption step, so that they decrypt the correct wire label the first time.
When garbling a gate, the garbler randomly assigns a select bit $0$ or $1$ to each wire label of the gate's input wires, where wire lables of the same wire have opposite bits.
More formally, $b$ is randomly sampled from $\{0,1\}$.
Then $A_0$ gets select bit $b$ and $A_1$ gets select bit $1-b$.
The garbler permutes the garbled table based on the select bits, and appends the select bits each wire label.
When the evaluator ungarbles the gate, they use the select bits appended to each of the two input wire labels to determine which row to decrypt.
For example, if the select bits on the end of the input wire labels are 1 and 0, then the evaluator decryps the third row.

Tables \ref{tbl:point-and-permute} show an example of point and permute.
$A, B$ and $C$ are wires, where $A_0$ and $A_1$ are the zeroith and first wire label of wire $A$ respectively.
The left table shows the select bits of the input wires $A$ and $B$.
The garbler gives $A_0$ select bit 0, determining that $A_1$ has select bit 1.
Likewise, the garbler gives $B_0$ select bit 1, determining that $B_1$ has select bit 0.

The garbler then permutes the garbled table based on the select bits. 
The permuted table is shown on the left in table \ref{tbl:point-and-permute}.
When evaluating this gate, the evaluator has $A_*$ and $B_*$ with select bits $a$ and $b$.
The evaluator decrypts the ciphertext in the row corresponding $a$ and $b$.

Intuitively, point and permute is secure because the select bits are independent of the truth value (also known as semantic value) of a the wire.
Thus it is secure for the garbler to permute the table based on the select bits, and it secure for the garbler to send the select bits to th evaluator.

Point and permute slightly increases garbler-side computation to substantially decrease evaluator-side computation.
The garbler samples $4$ additional random bits, and the evaluator performs a single decryption.
Without point and permute, the evaluator needs to decrypt $2.5$ ciphertexts on average, hence the garbler performs roughly $1.5$ fewer decryptions per gate.
The overall bandwidth is increased by $4$ bits per gate: a tiny constant increase. \footnote{The value is constant in the sense that it is independent of the security parameter.}

\al{Is this not an XOR/And gate? Make it one of them}

\begin{table}
    \label{tbl:point-and-permute}
    \centering
    \begin{tabular}{|c|c|}
        \hline
        Select Bit & Wire Label \\
        \hline
        0 & $A_0$ \\
        1 & $A_1$ \\
        1 & $B_0$ \\
        0 & $B_1$ \\
        \hline
    \end{tabular}
    \qquad
    \begin{tabular}{|c|c|}
        \hline
        Select Bits & Encryption \\
        \hline
        (0,0) & $\Enc_{A_0, B_1}(C_1)$ \\
        (0,1) & $\Enc_{A_0, B_0}(C_0)$ \\
        (1,0) & $\Enc_{A_1, B_1}(C_0)$ \\
        (1,1) & $\Enc_{A_1, B_0}(C_0)$ \\
        \hline
    \end{tabular}
    %\qquad
    %\begin{tabular}{|c|}
    %    \hline
    %    $C_0 \gets \{0,1\}^n$ \\
    %    $C_1 \gets \{0,1\}^n$ \\
    %    \hline
    %\end{tabular}
    \caption{Garbled Gate for Point and Permute}
    %\caption{Example garbled gate using point and permute. The gate being computed is given in figure \textbf{Make it 23:30 in mike's talk}}
\end{table}

\section{Garbled Row Reduction 3}
\al{check notation for get/sample in figure. $\gets$ is used for both}
Garbled Row Reduction 3 (GRR3) reduces the size of the garbled table from 4 ciphertexts to three 3 ciphertexts.
In a classical garbled circuit scheme, the wire labels for each wire are chosen prior to garbling any gates.
In GRR3, the garbler samples values for the input wire labels prior to garbling, and the garbler gives all other wire labels values as they generate each garbled table.

Suppose the garbler is garbling an XOR gate with input wires $A$ and $B$ and output wire $C$.
The garbler begins by using the point and permute method, sampling select bits and permuting the garbled table.
In GRR3, the garbler then set the ciphertext in the top row of the garbled table equal to a value that decrypts to $0^n$, the string of $n$ zeros.
Specifcally, $C_*$, the wire label on the top row, is set to $\EncInv_{A_*, B_*}(0^n)$.
The garbler sends the bottom three rows of the garbled table to the evaluator.
When evaluating the garbled gate, if the evaluator sees that the select bits of the input wires indicate to decrypt the first row, then the evaluator simply assumes the row to have value $0^n$ and decrypts it as usual.
In that case, the garbler set $C_*$ to $\EncInv_{A_*, B_*}(0^n)$.
Otherwise if the select bits indicate to decrypt the second, third or fourth row, the garbler decrypts the row as per usual.

Tables \ref{tbl:grr3} gives an example of garbling an XOR gate.
The left table shows the select bits of wire labels $A_0, A_1, B_0$ and $B_1$.
The right table shows the garbled table, in which the top row, the row associated with select bits $(0,0)$, is missing, as the row is assumed to have value $0^n$.
The bottom table shows the values of $C_0$ and $C_1$.
The value of $C_0$ is randomly sampled from $\{0,1\}^n$.

\begin{table}
    \centering
    \begin{tabular}{|c|c|}
        \hline
        Select Bit & Wire Label \\
        \hline
        0 & $A_0$ \\
        1 & $A_1$ \\
        1 & $B_0$ \\
        0 & $B_1$ \\
        \hline
    \end{tabular}
    \qquad
    \begin{tabular}{|c|c|}
        \hline
        Select Bits & Encryption \\
        \hline
        (0,1) & $\Enc_{A_0, B_0}(C_0)$ \\
        (1,0) & $\Enc_{A_1, B_1}(C_0)$ \\
        (1,1) & $\Enc_{A_1, B_0}(C_0)$ \\
        \hline
    \end{tabular}
    \qquad
    \begin{tabular}{|c|}
        \hline
        $C_0 \gets \{0,1\}^n$ \\
        $C_1 \gets \Enc_{A_0, B_1}^{-1}(0^n)$ \\
        \hline
    \end{tabular}
    \caption{Example garbled gate using point and permute and garbled row reduction 3. The gate being computed is given in figure \textbf{Make it 23:30 in mike's talk}}
    \label{tbl:grr3}
\end{table}

In considering the security of GRR3, we consider the effect of always setting one of the wire labels, $C_*$, to $0^n$.
Since the evaluator does not know whether $C_0$ or $C_1$ is set to $0^n$, the evaluator does not learn any information about the semantic representation of $C_*$..

GRR3 offers good performance benefits.
Garbler-side computation is the same, except that a decryption is performed in place of an encryption in constructing the first row of the garbled table.
Evaluator-side computation is the same as point and permute: the evaluator performs a single decryption.
Finally, GRR3 reduces the size of the garbled table from $4$ ciphertexts to $3$ ciphertexts, contributin a $25\%$ reduction in bandwidth.

\section{Free XOR}
The Free XOR technique makes the computation of XOR gates free, in the sense that no garbled table need to be communicated.
The evaluator can compute $C_*$ from only $A_*$ and $B_*$.

Like GRR3, the free xor techniques takes advantage of carefully crafted wire labels, even input wires.
To start, the garbler randomly samples a ciphertext $\Delta$ from $\{0,1\}^n$.
For each input wire $A$, let $A_0$ be randomly sampled from $\{0,1\}^n$ as before, and set $A_1 = A_0 \oplus \Delta$.

If the garbler is garbling an XOR gate, then the garbler does not constructed a garbled table or use select bits.
The garlber sets $C_0 = A_0 \oplus B_0$ and sets $C_1 = C_0 \oplus \Delta$.

The evaluator, when evaluating an XOR gate, simply computes $C_* = A_* \oplus B_*$.
As simple as it is, the evaluator will always acquire the correct value for $C_*$ based on the semantic value of $A_*$ and $B_*$.
The math for each of the four cases is shown:
\begin{align*}
    A_0 \oplus B_0 & = C_0 \\
    A_0 \oplus B_1 & = A_0 \oplus (B_0 \oplus \Delta) = (A_0 \oplus B_0) \oplus \Delta = C_1 \\
    A_1 \oplus B_0 & = (A_0 \oplus \Delta) \oplus B_0 = (A_0 \oplus B_0) \oplus \Delta = C_1 \\
    A_1 \oplus B_1 & = (A_0 \oplus \Delta) \oplus (B_0 \oplus \Delta) = (A_0 \oplus B_0) = C_0
\end{align*}
For any wire $A$, since $A_1$ is dependent on $A_0$, we often simplify notation such that the wire labels for $A$ are $A$ and $A \oplus \Delta$, omitting the subscript.

Free XOR is compatible with point and permute and GRR3; however, since XOR does not require a garbled table, GRR3 is only used on AND gates.

One interesting implication of using the Free XOR technique is that an added assumption must be made to our encryption algorithm.
Since $\Delta$ is part of the key and part of the the payload\footnote{The payload is the value that is being encrypted} of the encryption algorithm, the encryption algorithm must be secure under the circularity assumption.
Fortunately, the popular encryption scheme AES-128 is presumed to be secure under the circularity assumption.

Free XOR dramatically reduces bandwidth.
But since XOR gates are much cheaper than AND gates, circuits with more XOR gates perform faster.
Many programs have been made to optimize the number of XOR gates and minimize the number of AND gates (while minimizing the size of the entire circuit of course).
The garbler-side computation is reduced: constructing the xor garbled table does not require 3 encryptions and 1 decryption.
Evaluator-side computation is reduced likewise: xor gates do not requre any decryption.

\section{Garbled Row Reduction 2}
\al{https://web.engr.oregonstate.edu/~rosulekm/scbib/index.php?n=Paper.PSSW09}
Garbled row reduction 2 (GRR2) reduces the size of the garbled table of AND gates to $2$ ciphertexts.
Unfortunatley, GRR2 is not compatible with Free XOR, and GRR2 is less efficient than Free XOR combined with GRR3 for most circuits, so GRR2 is not often used by itself in practice.

Suppose the garbler is garbling an AND gate with input wires $A$ and $B$ and output wire $C$.
For all $a,b \in \{0,1\}$, let $V_{a,b} = H(A_a, B_b)$.
Since the evaluator acquires $A_a$ and $B_b$ for specific $a$ and $b$, they are also able to acquire $V_{a,b}$ for the same specific $a$ and $b$.
The garbler constructs a polynomial $P$ to be the unique 2-degree polynomial passing through the points $(1, V_{0,0}, (2, V_{0,1})$ and $(3, V_{0,1})$ - these $V$'s are selected because the garbler is garbling an AND gate.
And they set $C_0$ to $P(0)$.
The garbler next constructs a second polynomial $Q$ to be the unique 2-degree polynomial passing through points $(4, V_{1,1}), (5, P(5))$ and  $(6, P(6))$.
The garbler then sets $C_1$ to $Q(0)$.
The garbler sends $(5, P(5))$ and $(6, P(6))$ to the evaluator.

When de-garbling, the evaluator holds $a, b, A_a, B_b$ for some $a,b \in \{0,1\}$, $(5, P(5))$ and $(6, P(5))$.
With this information, the evaluator constructs polynomial $R$ to be unique the degree-2 polynomial passing through points $(2a + b + 1, H(A_a, B_b), (5, P(5))$ and $(6, P(6))$.
Polynomial $R$ is either $P$ or $Q$, depending on the values of $a$ and $b$.
The evaluator simply sets $C_*$ to be $R(0)$.

GRR2 is incompatible with Free XOR since it sets $C_0$ and $C_1$ implicitly and unpredictably.
Free XOR requires that $C_0$ = $C_1 \oplus \Delta$ for some global delta, which is not achievable with the randomly constructed polynomials.

\al{discuss security. Why is free xor secure?}
GRR2 alters garbler-side computation by having the garbler construct two polynomials instead of encrypting ciphertexts.
This is slightly more computation than required by FreeXOR and GRR3, but not an undermining amount.
Evaluator side computation increases slightly, as the evaluator constructs a polynomial instead of of decrypting two ciphertexts.


\section{FleXOR}
\al{passive tone}
After the creation of GRR2, secure computation was at an awkward point.
Circuits with many XOR gates were comptuted most quickly with Free XOR and GRR3, but circuits with many AND gates were computed most quickly with GRR2.
FleXOR reconciles GRR2 with Free XOR, resulting in a scheme that is universally faster.

Recall that GRR2 is incompatible with Free XOR because the wire labels are uncontrollably created by random polynomials, where Free XOR requires that $C_1 = C_0 \oplus \Delta$ for some global delta.
FleXOR solves this problem in a straightforward fashion: correct the delta value of output wires of AND gates such that the output wires use the global delta.
To correct the value, FleXOR adds a unary gate after each AND gate that corrects the wire label. \footnote{A unary gate is a gate that takes a single input wire and outputs a single wire. The unary gate does not change the semantic meaning of a wire label - that is, whether it represents 0 or 1. The unary gate merely alters the actual value of the wire label or ciphertext.}

Suppose an XOR gate has input wires $A$ and $B$ and output wire $C$.
Input wires $A$ and $B$ each come from an AND gate, so their labels are the result of the polynomial interpolation of GRR2.
$A$ has labels $A$ and $A \oplus \Delta_A$ and $B$ has labels $B$ and $B \oplus \Delta_B$.
To perform Free XOR, $A,B$ and $C$ need to be using the same delta value. 
The garbler adds an extra gate between $A$ and $B$ and the $XOR$ gate that adjusts their XOR value to the correct value.
$A$ and $A \oplus \Delta_A$ change to $A'$ and $A' \oplus \Delta$.
$B$ and $B \oplus \Delta_B$ change to $B'$ and $B' \oplus \Delta$.
Since $A, B$ and $C$ have the same delta value, FreeXOR is used.

The unary gate maps $A,A \oplus \Delta_1 \to A', A' \oplus \Delta$, where $\Delta$ is the correct delta value for the XOR gate.

FleXOR is made more efficient by not correcting the outputwire of every AND gate.
For example, if an output wire of an AND gate is immediately inputted into another AND gate, the wire label does not need to corrected.
Moreover, the wire labels do not even need to be corrected to a global delta.
Free XOR only requires that the three wires involved in the XOR gate, $A, B$ and $C$, use the same delta, so each XOR gate has its own $\Delta$.
For example, $A$ and $C$ may share a delta but $B$ may have a different delta, so it is sufficient to only correct $B$'s delta value.

FleXOR is fastest when AND gates are grouped together and XOR gates are grouped together, since fewer unary gates will be required.
Thereby, FleXOR creates an optimization problem: place XOR gates, AND gates and unary gates to minimize bandwidth requirements.
However, FleXOR's optimzation problem is too computationally expensive, but on the up side, analysis reveals that FleXOR requires on average an extra $0$ or $1$ ciphertext per gate, a relatively small amount.
That cost comes at the benefit of a Free XOR circuit, and 1 fewer ciphertext for each AND gate.

FleXOR requires slightly more garbler-side computation than FreeXOR and GRR2, since the garbler must create the unary gates.
The evaluator-side computation is the same as FreeXOR and GRR2, with the additional computation of the unary gates, which is small.
The size of the XOR garbled table is $0$, and the size of the AND garbled table is $2$.
But there is the additional communication of the unary gates' garbled tables.
The garbled table of the unary gate has $2$ ciphertexts, and the number of unary gates depends on the circuit.

FleXOR is intuitively secure, since the only additional information beyond GRR2 and FreeXOR is the unary gates.
The unary gate is secure, since it functions the same as a normal garbled gate except that an input is missing.
However, FreeXOR requires the circularity assumption, 
The circularity assumption is discussed in the section on FreeXOR.

\section{Half Gates}
Half Gates is the most recent improvement to garbled circuits.
The goal of Half Gates is to make AND gates cost two ciphertexts, while preserving properties necessary for Free XOR without adding unary gates. 
Half gates work by splitting an AND gate into two half gates.
Each half gate is an AND gate, where one of the inputs is \textit{known} to a party.
One half gate is the garbler-half-gate, where the garbler \textit{knows} the semantic value of one of the input wires.
Similarly, the evaluat knows the semantic value of one of the input wires of the evaluator-half-gate.
We then combine the garbler-half-gate and the evaluator-half-gate to produce a generic AND gate.

\al{where do the hashes come from?}
We first examine the garbler-half-gate.
Imagine an AND gate with input wires $A$ and $B$ and output wire $C$, where the garbler knows the semantic value of $A$.
This means that the garbler knows whether $A_*$ is $A_0$ or $A_1$.
Let $a,b$ and $c \in \{0,1\}$ represent the semantic value of $A$, $B$ and $C$ respectively.
If $a = 0$, then $c = a \wedge b = 0$, so the garbler creates a unary gate that always outputs $C_0$.
If $a = 1$, then $c = a \wedge b = b$, so the garbler creates a unary identity gate that outputs $C_b$.
We combine the unary gates into a single garbled table, shown in table \ref{tbl:halfgate-gg-garb}.
Since the evaluator has $B_*$, the evaluator can compute the label of the output wire of the AND gate by xoring the rows in the garbled table by $H(B)$ or $H(B \oplus \Delta)$ to get $C$ or $C \oplus \Delta$ respectively. 

The GRR3 trick reduces the size of the garbler-half-gate garbled table to $1$ ciphertext.
Choose $C$ such that the first of the top row of the garbled table is the all zeros ciphertext. 
The top row may not necessarily be $H(B) \oplus C$, since for security the rows are permuted. 
Because the top row is all 0s, it does not need to be sent to the evaluator. 
If the evaluator should decrypt the cipher text on the top row (as directed by point and permute), then the evaluator assumes the cipher text to be all 0s. Overall, computing $a \wedge b = c$ requires two having operations by the generator, a single hash operation by the evaluator, and the communication of one ciphertext. 

\begin{table}[]
    \label{tbl:halfgate-gg-garb}
    \centering
    \begin{tabular}{|c|}
        \hline
        Garbled Table for $a = 0$ \\
        \hline
        $H(B) \oplus C$ \\
        $H(B) \oplus C$ \\
        \hline
    \end{tabular}
    \begin{tabular}{|c|}
        \hline
        Garbled Table for $a = 1$ \\
        \hline
        $H(B) \oplus C$ \\
        $H(B) \oplus C \oplus \Delta$ \\
        \hline
    \end{tabular} $\;\rightarrow$
    \begin{tabular}{|c|}
        \hline
        Garbled Table for any $a$ \\
        \hline
        %$\Enc_B(C)$ \\
        $H(B) \oplus C$ \\
        $H(B) \oplus C \oplus a\Delta$ \\
        \hline
    \end{tabular}
    \caption{Generator's Garbled Half Gate for $a = 0$, $a = 1$, and written more succinctly with $a\Delta$ for $a \in \{0,1\}$. If $a = 0$, then $a\Delta = 0$.  Otherwise if $a = 1$, then $a\Delta = \Delta$.}
\end{table}

\al{finished up to here, sort of}

We now consider computing $a \wedge b = c$, where the evaluator somehow already knows the value of $a$.
If $a = 0$, then the evaluator should acquire $C$. 
Otherwise if $a = 1$, then the evaluator should acquire $C \oplus b\Delta$, in which case it is sufficient for the evaluator to obtain $\Omega = C \oplus B$ (then xor $\Omega$ with the wire label corresponding to $b$).
Table \ref{tbl:halfgate-gg-eval} shows cipher texts which the garbler gives to the evaluator.

% explained at 40:00
This table is different from other garbled tables. 
First, the table does not need to be permuted. 
Secondly, evaluation is different.
If $a = 0$, then the evaluator uses wire label $A$ to decrypt the top row of the table and acquire $C$. 
If $a = 1$, then the evaluator uses $A \oplus \Delta$ to decrypt the second row, yielding $C \oplus B$.
The evaluator then xors wire label $B + b\Delta$ by $C \oplus B$ to obtain $C \oplus b\Delta$. 
As before, the ciphertext  on the top row does not need to be communicatd by using the garbled row-reduction trick.
The generator sets $C$ such that $H(A) \oplus C = 0$ (i.e. $C = H(A)$). 
The total cost of this half gate is the same as before: two hashes by the generator, one hash by the evaluator, and once cipher text. 

\begin{table}[h]
    \label{tbl:halfgate-gg-eval}
    \centering
    \begin{tabular}{|c|}
        \hline
        Garbled Table for any $A$ \\
        \hline
        $H(A) \oplus C$ \\
        $H(A \oplus \Delta) \oplus C \oplus B$ \\
        \hline
    \end{tabular}
    \caption{Evaluator's half gate garbled table.}
\end{table}

We now put the two half gates together to form an AND gate. 
Consider the following where $r$ is a random bit generated by the generator:
\begin{equation}
    a \wedge b = a \wedge (r \oplus r \oplus b) = (a \wedge r) \oplus (a \wedge (r \oplus b)).
\end{equation}
The first AND gate, $a \wedge r$, can be computed with a generator-half-gate - the generator ``knows'' $r$. 
Furthermore, if we can let the evaluator know the value of $r \oplus b$, then the second AND gate, $(a \wedge (r \oplus b)$, can be computed with an evaluator-half-gate - the evaluator ``knows'' $r \oplus b$. 
And the final XOR can be computed with free xor at the cost of no ciphertexts. 

It is secure for the garbler to give $r \oplus b$ to the evaluator, since $r$ is random and blinds the value of $b$. 
The value of $r \oplus b$, while only a single bit, can be communicated to the evaluator for free: use the select bit (from the point and permute technique) of the false wire label on wire $b$ (so $r$ is the select bit on the true wire label of wire $b$).

The overall cost of using Half Gates for AND gates is four calls to $H$ for the generator, two calls to $H$ for the evaluator, and the communication of $2$ ciphertexts. 
Half Gates guarantees only two ciphertexts are needed per AND gate, but the tradeoff is the additional computation for both parties (i.e. computing $H$).
With FleXOR, the number of ciphertexts that need to be communicated may vary, but there is less computation required.

When secure computation becomes used in real operations, the circuit will likely not be optimized for XOR gates (reducing the number of AND gates). 
In this case, where an arbitrary function is being computed, it is hypothesized that Half Gates will outperform FleXOR. 
But in many of the cases being examined now - most notably computing AES - the circuit has been optimized heavily to have very few AND gates, hence the benefits of Half Gates are less noticeable.\footnote{At present, the AES circuit is 80\% XOR gates} 

\section{Improving Oblivious Transfer}
\al{Fix this up}
This section discusses improvements to oblivious transfer, the method by which the garbler communciates the wire labels corresponding to the evaluator's inputs to the evaluator.
At its most basic, oblivious transfer enables Alice to potentially send one of two messages to Bob.
Bob selects which message he wants to recieve.
The desirable securitiy properties are (1) Alice does not konw which message she sent to Bob and (2) Bob cannot infer any informatoin about the message that he did not recieve.

Oblivious transfer is a key component in secure computation, and often consumes a substantial portion of the secure computation protocol's time.
There are two major improvements to oblivious transfer.
The first is called \textit{OT-extension}.
When using OT with a garbled circuit scheme, Alice and Bob are not exchaning just one message: they exchange a wire label for each of Bob's inputs.
This means that Alice and Bob are performing a linear number of OTs based on the number of inputs.
OT-extension reduces the entire OT phase to a constant number of OT operations.
In particular, Alice and Bob run a constant number of OT in order to generate a polynomial number of exchanged messages.

The second improvement to OT is called \textit{OT-preprocessing}.
In OT-preprocessing, Alice and Bob performs the expensive OT operation ahead of time, in what is called the offline phase, on random values.
Then, during the online phase, Alice and Bob quickly exchange values to correct the pre-shared messages.
Alice and Bob only exchange a single message during the online phase, and there is no computationally expensive math required like there is during the real OT operation.

In our work with secure computation, OT-preprocessing substantially improved the performance of garbled circuits, and OT-extension was less useful.
If a garbled circuit scheme uses an offline phase and thereby uses OT-preprocessing, then OT-extension is largely unneccessary.
The parties take the extra needed to send all of the wire labels in the offline phase, since the time does not matter.
What we the parties are truly concerned about is online time.
During online time, OT-preprocessing offers huge advantages.

