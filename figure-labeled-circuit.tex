%!TEX root = thesis.tex
\begin{figure}[h]
    \center

\begin{circuitikz} \draw
% http://tex.stackexchange.com/questions/55213/how-to-draw-a-boolean-circuit-diagram-in-circuitikz
% http://texdoc.net/texmf-dist/doc/latex/circuitikz/circuitikzmanual.pdf
% adapted from figure on page 40
(0,2) node[and port] (myand1) {}
(0,0) node[and port] (myand2) {}
(3,1) node[xor port] (myxor) {}
(myand1.in 1) node[left=.8cm](a) {$x_0$}
(myand1.in 2) node[left=.8cm](b) {$x_1$}
(myand2.in 1) node[left=.8cm](c) {$y_0$}
(myand2.in 2) node[left=.8cm](d) {$y_1$}
(myxor.out) node[right=.5cm](e) {$z_0$}
(myxor.out) node[right=.25cm](f) {}
(a) -| (myand1.in 1)
(b) -| (myand1.in 2)
(c) -| (myand2.in 1)
(d) -| (myand2.in 2)
(myand1.out) -| (myxor.in 1)
(myand2.out) -| (myxor.in 2)
(myxor.out) -| (f);

% classic tikz
\node at (-.7,2.0) {$G_0$};
\node at (-.7,0.0) {$G_1$};
\node at (2.4,1.0) {$G_2$};
\node at (-1.7,2.5) {$W_0$};
\node at (-1.7,2.0) {$W_1$};
\node at (-1.7,0.5) {$W_2$};
\node at (-1.7,0.0) {$W_3$};
\node at (0.7,2.3) {$W_4$};
\node at (0.7,0.3) {$W_5$};
\node at (3.35,1.25) {$W_6$};

\end{circuitikz}
\caption{A simple boolean circuit with labeled inputs, output, gates and wires.}
\label{fig:labeled-circuit}
\end{figure}

