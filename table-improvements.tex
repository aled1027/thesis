\begin{table}[t]
    \centering
    \renewcommand{\arraystretch}{1.2}
    \normalsize
    \begin{adjustbox}{width=1\textwidth}
        \begin{tabular}{|p{5cm}|c|c|c|c|c|c|}
            \hline
            \multirow{2}{5cm}{\centering \textbf{Garbled Circuit Improvement}} & 
            \multicolumn{2}{c|}{\textbf{Table Size (x$\lambda$)}} & 
            \multicolumn{2}{c|}{\textbf{Garble Cost}} & 
            \multicolumn{2}{c|}{\textbf{Eval Cost}} \\
            \cline{2-7}
            & \textbf{XOR} & \textbf{AND} & \textbf{XOR} & \textbf{AND}  & \textbf{XOR} & \textbf{AND} \\
            \hline
            Classical & 4 & 4 & 4 & 4 & 4 & 4 \\ \hline
            Point and Permute & 4 & 4 & 4 & 4 & 1 & 1 \\ \hline
            GRR3 & 3 & 3 & 4 & 4  & 1 & 1 \\ \hline
            Free XOR & 0 & 3 & 0 & 4 & 0 & 1  \\ \hline
            GRR2  & 2 & 2 & 4 & 4 & 1 & 1  \\ \hline
            FleXOR & \{0,1,2\} & 2 & \{0,2,4\} & 4 & \{0,1,2\} & 1  \\ \hline
            Half Gates & 2 & 0 & 2 & 0 & 0 & 2  \\ \hline
        \end{tabular}
    \end{adjustbox}
    \caption[Summary of garbled circuit improvements]{Summary of Garbled Circuit Improvements. 
    GRR3 stands for garbled row reduction three and GRR2 stands for garbled row reduction two. 
    The values shown for each improvement include benefits from point and permute and other compatible improvements: 
    Free XOR uses GRR3; FleXOR uses GRR2 and Free XOR; and Half Gates use FreeXOR.
    This table is adapted from \cite{twohalves}.}
    \label{tbl:improvements}
\end{table}


