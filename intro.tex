%!TEX root = thesis.tex


%The \introduction command is provided as a convenience.
%if you want special chapter formatting, you'll probably want to avoid using it altogether
\chapter*{Introduction}
     \addcontentsline{toc}{chapter}{Introduction}
\chaptermark{Introduction}
\markboth{Introduction}{Introduction}

Suppose that millionaires Alice and Bob wish to determine who is wealthier, but they do not want to disclose their wealth to each other.
Alice and Bob could tell a trusted third party their wealth, and then that trusted third party could tell Alice and Bob who is wealthier.
However, that method has many disadvantages, one being that they need to trust a third party.
What if there were a method that allows Alice and Bob to determine who is wealthier by only sending messages between themselves?

This classic problem is known as the \textit{millionaire problem}, and it is a special case of \textit{secure computation}.
Secure computation gives multiple parties the ability to jointly compute an arbitrary function, while keeping each party's input private from all other parties.
In the millionaire problem, there are two parties, Alice and Bob, and they wish to compute the \textit{less than} function.

We generalize secure computation to allow for an arbitrary number of parties to compute an arbitrary function amongst themselves. 
By generalizing secure computation, we open up many possible uses where secure computation can improve an existing process or allow for computations that are otherwise impossible or illegal. 
For example, we can run an election without a centralized body; we can perform an auction without using an auctioneer; and we can enable multiple hospitals with sensitive healthcare data to work together without sacrificing the privacy of their data.
While secure computation works with arbitrarily many parties, this thesis focuses on the special case where there are two parties, called two-party computation (2PC).

The most common method for performing 2PC is \textit{garbled circuits}.
In a garbled circuit protocol, the parties transform their function $f$ into a circuit. 
They then \textit{garble} their circuit, such that with a bit more information they can \textit{evaluate} the garbled circuit, recovering the answer to their function.
Garbling the circuit involves obscuring the truth value of each wire inside the circuit.
Once the truth values are obscured, the parties cannot learn any information as they process, or evaluate, the circuit: they only learn the answer to their function.

Garbled circuits have been heavily researched and optimized since their creation in the late 1980s.
For most of that period, garbled circuit protocols were too slow and cumbersome for usage in the real world, but as of late, the protocols have achieved speeds comparable to that of loading a webpage. 

In order to further increase the speed of garbled circuits, researchers split the garbled circuit protocol into two phases, an offline phase and an online phase.
In the offline phase, before the two parties determine their inputs to the function, the parties exchange as much information as feasible. 
Later, when the parties decide to compute the function, and they determine determine their inputs to the function, the parties engage in an online phase, where they exchange more information and finish the garbled circuit protocol, whereby they recover the output to their function.
Using an offline/online protocol greatly reduces the latency of the computation - the time that it takes to go from determining inputs to acquiring an answer.

The offline/online system, while offering a number of benefits, is also limiting in that the function is determined ahead of time in the offline phase.
Say that millionaires Alice and Bob originally agree to find out who is wealthier, so they exchange information for that function ahead of time in the offline phase, but then at the beginning of the online phase, Alice and Bob change their minds. 
Instead, they now wish to verify that they are indeed both millionaires, and verify that the other party is not going bankrupt.
Since they exchanged information specific to the ``who is wealthier'' function during the offline phase, they are stuck. 
They cannot pivot and compute the new function without large computational sacrifices.

This thesis presents a new method called \textit{component-based} garbled circuits to solve this problem.
In the offline phase, instead of exchanging information corresponding to a single function, Alice and Bob exchange smaller pieces of information that can be stitched together to compute a class of functions.
For example, instead of exchanging a garbled circuit that computes the ``who is wealthier'' function, Alice and Bob exchange many garbled circuits which are \textit{components} or subparts of the \textit{less than} function and other similar functions.
Then, during the online phase, Alice and Bob select the function that they wish to compute from the class of available functions, and build their function by \textit{chaining} pre-exchanged components.
With component-based garbled circuits, Alice and Bob have more flexibility. 
They are no longer stuck using their original function - they can securely compute a host of functions at their whim. 

This thesis is part of a larger project which created component-based garbled circuits; specifically, it contributes two items to the project.
The first is a new method that improves the efficiency of component-based garbled circuits. 
The original component-based garbled circuits scheme requires that a ciphertext be communicated per wire chained - in other words, the communication scales linearly with the size of data.
Our new method, called \textit{single communication multiple connections} (SCMC), requires a single ciphertext be communicated per block of data instead of per wire. 
The communication requirement for chaining is now linear in the number of data objects but is constant in the size of the data object: chaining a $10$ by $10$ matrix has the same bandwidth requirement as chaining a $1,000$ by $1,000$ matrix. 
This allows for extremely fast computation of large statistical operations. 

The second, and larger, contribution of this thesis is the implementation of component-based garbled circuits and SCMC into a program called \CompGC. 
\CompGC is a full-fledged secure computation system, where parties connect via TCP to agree on a function, exchange inputs, and securely compute the function.
The program is fast, beating the best timings in the literature even for functions that do not benefit the most from SCMC.

For those that are not familiar with cryptography, it is best to read Chapters 1, 2 and 3 as they provide the necessary background.
Chapter 1 introduces cryptographic primitives, and Chapter 2 discusses what it means for a secure computation protocol to be secure and introduces garbled circuits.
Chapter 3 builds off of Chapter 2 by explaining a variety of improvements to garbled circuits.
Chapter 4 and 5 focus on component-based garbled circuits.
Chapter 4 explains the basic idea of component-based garbled circuits, and gives our improvement, SCMC.
Chapter 5 discusses our implementation of component-based garbled circuits and SCMC, \CompGC, and gives performance metrics of \CompGC.







