%!TEX root = thesis.tex

\chapter*{Abstract}

Secure computation is a cryptographic method for securely computing a function between two parties while preserving the privacy of the parties' inputs.
Some modern research into secure computation uses an offline/online scheme, where parties pre-process and exchange information in an offline phase prior to computing the function in an online phase. 
The offline/online setting has many advantages, namely that the online computation is very fast, but the setting is limiting, as the function that is being computed must be determined ahead of time.

This thesis is part of a larger project that created component-based garbled circuits to address some of the problems with the offline/online setting.
Component-based garbled circuits observe that many real-world functions are composed of standard components such as arithmetic operations, matrix operations and other common operations.
Component-based garbled circuits exchange many small, generic garbled components in an offline phase. 
Later in an online phase, the parties choose a function they wish to compute, and they stitch together the garbled components in order to create the function.

This thesis contributes two items to the larger project of component-based garbled circuits.
The first item is \textit{Single Communication Multiple Connections} (SCMC); SCMC is a technique that improves the online bandwidth of component-based garbled circuits.
The second item is an implementation, \CompGC, of component-based garbled circuits.
\CompGC is designed to be fast and secure; we run performance measurements on \CompGC, and we find that component-based garbled circuits offer considerable savings in online time and online bandwidth over other implementations of garbled circuits. 
